\section{Electrical charge carriers and electric fields}

\begin{multicols}{2}
	\begin{itemize}
		\item State that there are 2 types of charge: positive (+) and negative (-)
		\item State that charges that are the same are called like charges
		\item State that charges that are not the same are called opposite charges
		\item State that like charges repel each other
		\item State that opposite charges attract each other
		\item Describe a simple model of an atom that includes: protons (+), electrons (-) and neutrons (0)
		\item Be able to draw electric fields for; point charges, pairs of charges, parallel plates.
		\item State that in an electric field, charged objects and particles experience a force
		\item Carry out calculations involving the relationship between charge, current and time
		\item State that electrons are free to move in a conductor
		\item State that electrical current is a flow of charges around a circuit or through a conductor
	\end{itemize}
\end{multicols}

\section{Potential difference (voltage)}
\begin{multicols}{2}
	\begin{itemize}
		\item State that the voltage of a supply is a measure of the energy given to each charge in the circuit
		\item State that hte units of voltage and potential difference are the volt(B) or joules per coulomb
		\item State that like charges repel eachother
		\item State that opposite charges attract eachother
		\item Explain both DC and AC in terms of current and voltage
	\end{itemize}
\end{multicols}

\section{Practical electrical and electronic circuits}
\begin{multicols}{2}
	\begin{itemize}
		\item Draw and identify circuit symbols for: a cell, battery, resistor, fuse, switch, lamp, ammeter and voltmeter in a circuit
		\item Draw circuit diagrams to show the correct positions of an ammeter and voltmeter in a circuit
		\item State that in a series circuit current is the same at all points
		\item State that the sum of the voltages across the components in a series circuit is equal to the supply voltage
		\item State that the sum of the currents in a parallel circuit is equal to the supply current
		\item State that the potential difference (voltage) across components connected in parallel is the same for each component
		\item Carry out calculations involving resistors connected in series 
	\end{itemize}
\end{multicols}

\section{Ohm's law}
\begin{multicols}{2}
	\begin{itemize}
		\item State that V/I for a resistor remains constant for different currents
		\item 
	\end{itemize}
\end{multicols}

\section{Electrical Power}
\begin{multicols}{2}
	\begin{itemize}
		\item State that power is the electrical energy transferred each second
		\item State that the power, $P = IV$
		\item Carry out calculations involving the relationships between power,
			energy, time, current and potential difference
		\item Explain the equivalence between $IV$, $I^2$ and $v^2 R$
		\item Carry out calculatins involving relationships between power, current,
			voltage and resistance.
	\end{itemize}
\end{multicols}

\section{Conservation of energy}
\begin{multicols}{2}
	\begin{itemize}
		\item Give examples of devices where conversion takes place and state the
			conversion
		\item State that the resistance of a thermistor decreases with increasing temperature
		\item State that the resistance of an LDR decreases with increasing light 
			level
		\item Carry out calculations involving $V=IR$ for a thermistor and LDR
		\item Be able to draw the symbol for an LED
		\item State that an LED only lights when current flows through it in a
			specific direction
		\item Draw a working circuit containing an LED
		\item State that the reesistor in series with an LED is there to limit the
			current to protect the LED
		\item Be able to draw an n-channel enhancement MOSFET
		\item State that transistors can be used as electronic switches to explain
			the operation of a transistor switching circuit
		\item State that work done is a measure of the energy transferred
		\item Carry out calculations involving the relationship between work done,
			force and distance
		\item Carry out calculations involving the relationship between change in
			gravitational potential energy, gravitational field stregnth, mass and
			height
		\item Carry out calculations involving the relationship between kinetic energy, mass and velocity
		\item Carry out calculations involving the relationship between work done,
			power and time
	\end{itemize}
\end{multicols}

\section{Specific heat capacity}
\begin{multicols}{2}
	\begin{itemize}
		\item Use the following terms correctly in context: temperature, heat, 
			celsius
		\item  State that the temperature of a substance is a measure of the average
			kinetic energy of the particles in the substance
		\item State that heat is transferred from high temperature objects to low
			temperature objects by: conduction, convection and radiation
		\item State that the heat loss every second from a hot object is dependent
			on the temperature difference between the object and its surroundings
		\item  State that the same mass of different materials requires different
			quantities of energy to change their temperature by 1 degree Celsius
		\item Carry out calculations involving: energy, mass, specific heat capacity
			and temperature change
		\item State that energy is gained or lost by a material when its state changes
		\item State that a change in state of a material does not involve a change in
			the materials temperature
		\item Carry out calculations involving energy, mass and specific latent heat
		\item Use the following terms correctly in context: specific heat capacity,
			change of state, latent heat of fusion, latent heat of vaporisation.
	\end{itemize}
\end{multicols}

\section{Gas laws and the kinetic model}
\begin{multicols}{2}
	\begin{itemize}
		\item State that pressure if the force per unit area, when a force acts at right
			angles to a surface.
		\item  State that 1 Pascal is equal to 1 Newton per metre squared
		\item  State that the pressure exerted by a fixed mass of gas at constant
			temperature is inversely proportional to its volume
		\item State that the pressure exerted by a fixed mass of gas at constant
		\item Carry out calculations involving pressure, force and area volume is
			directly proportional to its temperature measured in kelvins
		\item Describe how the kinetic model accounts for the pressure exerted by (K)
			a gas
		\item State that the volume of a fixed mass of gas at constant pressure is
			directly proportional to its temperature measured in kelvins (K)
		\item Carry out calculations to convert temperatures in 0C to K and vice versa
		\item State the ideal gas equation
		\item Carry out calculations involving pressure, volume and temperature
		\item of a fixed mass of gas using the general gas equation
		\item Explain what is meant by absolute zero of temperature
		\item Explain the pressure-volume, pressure-temperature and
			volume-temperature laws in terms of the kinetic model.
	\end{itemize}
\end{multicols}
