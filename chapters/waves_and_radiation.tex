\section{Wave parameters and behaviour}
\begin{multicols}{2}
	\begin{itemize}
		\item State that a wave is a transfer of energy
		\item State the difference between a transverse and longitudinal wave
		\item State that the Amplitude is a measure of a waves energy
		\item State that in a given medium wave speed is constant
		\item State that frequency of a wave is dependent on its source 
		\item Use the following terms correctly and in context: wave, crest, trough,
			frequency, wavelength, speed, amplitude, period
		\item Carry out calculations involving the relationship between frequency
			and period
		\item Carry out calculations involving the relationship between frequency,
			wavelength and speed
		\item State what is meant by diffraction
		\item Know how wavelength and gap width affect diffraction
	\end{itemize}
\end{multicols}

\section{EM spectrum}
\begin{multicols}{2}
	\begin{itemize}
		\item State in order of wavelength the members of the electromagnetic
			spectrum: Radio, television, microwaves, infra-red, visible light, ultra
			violet, x-rays, gamma rays
		\item State that all members of the electromagnetic spectrum are transverse
			waves.
		\item State that all members of the electromagnetic spectrum travel through
			a vacuum or air at a speed of 3x108m/s
		\item State that the energy of a photon in the electromagnetic spectrum is
			proportional to the photons frequency
	\end{itemize}
\end{multicols}

\section{Light}
\begin{multicols}{2}
	\begin{itemize}
		\item State what is meant by reflection of light 
		\item Draw diagrams showing reflection taking place
		\item Use in the correct context the terms: incident ray, reflected ray, 
			normal, angle of incidence, angle of reflection
		\item State what is meant by Total Internal Reflection
		\item  Draw diagrams showing total internal reflection taking place
		\item State a practical application for total internal reflection
		\item State what is meant by refraction of light
		\item Draw diagrams showing refraction from one medium to another
		\item Use in the correct context the terms: incident ray, refracted ray,
			normal, angle of incidence, angle of refraction
	\end{itemize}
\end{multicols}

\section{Nuclear radiation}
\begin{multicols}{2}
	\begin{itemize}
		\item Describe a simple model of an atom including protons, neutrons and
			electrons
		\item  State what is meant by the terms alpha particle, beta particle and
			gamma ray
		\item  State that radiation can be absorbed by materials
		\item State the approximate range through air and absorbers of alpha,
			beta and gamma radiation
		\item Explain the term ionisation
		\item State that alpha is the most ionising type of radiation
		\item Describe how detectors of radiations work
		\item State that radiation can kill living cells or change the nature of living
			cells
		\item State that absorbed dose is the energy absorbed per unit mass of the
			absorbing material
		\item Carry out calculations using the relationship between absorbed dose,
			energy absorbed and mass of absorber.
		\item State that a radiation weighting factor s given to each radiation as a
			measure of its biological effect
		\item State equivalent dose is the product of absorbed dose and radiation
			weighting factors
		\item Carry out calculations using the relationship between absorbed dose,
			weighting factor and equivalent dose.
		\item State that the equivalent dose rate is the equivalent dose per unit time
		\item Carry out calculations using the relationship between equivalent dose
			rate, equivalent dose and time
		\item State that the risk of biological harm from exposure to radiation
			depends on: the absorbed dose, the type of radiation, the type of
			tissue exposed
		\item Describe factors that affect background radiation
		\item Describe safety procedures for handling radioactive materials
		\item State that exposure to radiation is reduced by: shielding, limiting
			exposure time and increasing distance from the source.
		\item State one medical use of nuclear radiation
		\item State one non-medical use of nuclear radiation
		\item State that the activity of a source is how a measure of how many
			nuclei decay in 1 second
		\item Carry out calculation involving the relationship between, activity,
			number of nuclei decaying and time
		\item State that activity of a source decreases over time o State the meaning
			of the term half-life
		\item Carry out calculations involving half- life
		\item Describe a method for measuring the half life of a source
		\item State advantages and disadvantages of using nuclear power in the
			generation of electricity.
		\item Describe the process of Fission: A heavy nucleus splitting into 2
			lighter nuclei and releasing neutrons
		\item Explain in simple terms a chain reaction
		\item Describe the process of Fusion: 2 light nuclei combine to form 1
			heavier nucleus
		\item Describe the principles of operation of a nuclear reactor in terms of:
			fuel rods, moderator, control rods, coolant, containment vessel
		\item Describe the problems associated with the disposal and storage of
			nuclear waste.
	\end{itemize}
\end{multicols}
