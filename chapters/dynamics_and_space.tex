\section{Motion}
\begin{multicols}{2}
	\begin{itemize}
		\item What is meant by the term speed?
		\item Can I carry out calculations involving the relationship between
			distance, time and average speed (d=vt)?
		\item Can I carry out calculations involving the relationship between
			distance, time and average speed (d=vt)?
		\item Can I identify the scalar and vector quantities from the following?
			Force, speed, velocity, distance, displacement, mass, time and energy.
		\item What is the difference between distance and displacement?
		\item What is the difference between speed and velocity?
		\item Can I calculate the resultant of two vectors in:
		\item Can I describe how to measure an instantaneous speed? a) a straight line
		\item Can I calculate the instantaneous speed of an object? b) at right angles?
		\item Can I describe one example where the average speed of an object is
			measured in everyday life?
		\item Can I describe one example where the instantaneous speed of an
			object is measured in everyday life?
		\item Can I identify situations where average speed and instantaneous
			speed are different?
		\item What is meant by a scalar quantity?
		\item What is meant by a vector quantity?
		\item Can I carry out calculations involving the relationship between
			displacement, time and average velocity (s=vt)?
		\item What is meant by the term acceleration?
		\item Can I carry out calculations involving the relationship between initial
			velocity, final velocity, time and uniform (constant) acceleration?
		\item From a speed-time graph, can I identify when an object has:
			\begin{enumerate}[label=(\alph*)]
				\item increasing speed
				\item decreasing speed
				\item constant speed
			\end{enumerate}
		\item From a speed-time graph can I calculate the distance travelled by an
			object?
		\item Can I plot a velocity-time graph given a set of data?
		\item From a velocity-time graph, can I identify when an object has:
			\begin{enumerate}[label=(\alph*)]
				\item increasing velocity
				\item decreasing velocity
				\item constant velocity
			\end{enumerate}
		\item From a velocity–time graph involving more than one constant
			acceleration, can I calculate the acceleration of an object?
		\item From a velocity–time graph involving more than one constant
			acceleration, can I calculate the displacement of an object?
	\end{itemize}
\end{multicols}

\section{Forces}
\begin{multicols}{2}
	\begin{itemize}
		\item When a force is applied to an object, what effect will it have on it?
			(i.e. what will it change?)
		\item Can I describe how to measure a force using a Newton Balance?
		\item Can I define the Newton?
		\item Is force a vector or a scalar quantity?
		\item In which direction does friction act in relation to the motion of an
			object?
		\item Can I describe and explain situations in which attempts are made to
			increase or decrease the force of friction?
		\item What is meant by the term balanced forces? 
		\item Can I use free body diagrams to analyse the forces on an object?
		\item What is meant by the resultant of a number of forces?
		\item Can I explain how an object travels at a constant speed? (Think of
			Newton’s first law of motion and frictional forces)
		\item Can I apply Newton’s first law of motion to explain constant
			velocity?
		\item Can I predict what will happen to the acceleration of an object if only
			the mass changes?
		\item Can I predict what will happen to the acceleration of an object if only
			the force changes?
		\item Can I use the equation F=ma when only one force is acting?
		\item Can I use the equation F=ma when more than one force is acting?
		\item Can I use Newton’s laws to explain:
			\begin{enumerate}[label=(\alph*)]
				\item the motion of an object during free-fall and
				\item why it reaches terminal velocity?
			\end{enumerate}
		\item What is work done a measure of? 
		\item Can I carry out calculations involving the relationships between work
			done, force and displacement? ($E = F s$)
		\item What is weight an example of?
		\item What does weight mean?
		\item Do I know the difference between weight and mass and what is the unit
			of each quantity?
		\item Can I explain what is meant by gravitational field strength?
		\item Can I carry out calculations involving the relationship between weight,
			mass and gravitational field strength including situations where g is not
			equal to $10 \frac{N}{kg}$?
		\item Can I state Newton's 3rd law of motion?
		\item Can I apply Newton's 3rd law of motion to explain motion resulting from
			a reaction force?
	\end{itemize}
\end{multicols}

\section{Satellites and Projectiles}
\begin{multicols}{2}
	\begin{itemize}
		\item What is meant by the period of a sattelite?
		\item How does the period of a sattelite depend on the height of its orbit?
		\item How does the height of orbit of a geostationary sattelite compare with
			other sattelites?
		\item At what speed do radio (or microwave) signals travel during sattelite
			communication?
		\item  Can I use the relationship between distance, speed and time when
			applied to sattelite communications?
		\item Can I name at least 3 applications of sattelites?
		\item Can I describe how parabolic (curved) reflectors are used in sattelite
			communication to
			\begin{enumerate}[label=(\alph*)]
				\item transmit signals?
				\item recieve signals?
			\end{enumerate}
		\item Can I explain how sattelites have developed our understanding of the
			global impact of our actions?
		\item How can a sattelite be used to monitor environmental changes on the
			Earth?
		\item Can I explain how projectile motion can be treated as two independant
			motions?
		\item Can I carry out calculations of projectile motions using:
			\begin{enumerate}
				\item appropriate formulae?
				\item graphs?
			\end{enumerate}
		\item Can I explain how a sattelite orbits in terms of projectile motion?
	\end{itemize}
\end{multicols}

\section{cosmology}
\begin{multicols}{2}
	\begin{itemize}
		\item What is a star?
		\item What is a planet?
		\item What is a galaxy?
		\item What is a solar system?
		\item What is a moon?
		\item What is an exo-planet?
		\item What is the universe?
		\item What does a light year measure?
		\item How many metres are in 1 light year?
		\item Can I explain why we use the light year to measrue distance in space?
		\item Can I calclate the number of metres in 1 light year?
		\item What is the distance in light years from the earth to
			\begin{enumerate}[label=(\alph*)]
				\item The sun?
				\item The next nearest star?
				\item The next galaxy?
				\item The edge of the universe?
			\end{enumerate}
		\item What conditions are required for an exp-planet to sustain life?
		\item What is the name of the theory of the origin of the universe?
		\item Can I describe what happened when the universe began?
		\item What evidence is there to support the hot big bang model of the
			universe?
		\item How old do we think the universe is? What evidence is there to suggest
			the age of the universe?
		\item What is the electromagnetic spectrum?
		\item What do all the waves of the electromagnetic spectrum have in common?
		\item Can I list the waves of the electromagnetic spectrum in order of
			\begin{enumerate}
				\item Frequency?
				\item Wavelength?
			\end{enumerate}
		\item Can I name an example of a detector for each of the waves in the
			electromagnetic spectrum?
		\item Why have astronomers developed telescopes to detect different parts of
			the electromagnetic spectrum?
		\item What information have astronomers obtained from using these telescopes?
		\item Can I identify a continuous spectrum from a picture?
		\item Can I identify a line emission spectrum from a picture?
		\item Can I use a line spectrum to identify the elements present in stars?
		\item Do I know that radiation from space is recieved in a variety of forms?
	\end{itemize}
\end{multicols}

\section{Space exploration}
\begin{multicols}{2}
	\begin{itemize}
		\item What have we learned about planet Earth as a result of space exploration?
		\item What have we learned about the universe as a result of space exploration?
		\item How has our model of the universe changed over time/
		\item What evidence is there to support our understanding of the universe now?
		\item Can I apply Newton's second law ($F = ma$) to describe the motion and
			the forces acting on a space rocket during
			\begin{enumerate}[label=(\alph*)]
				\item launch?
				\item motion in space?
				\item landing?
			\end{enumerate}
		\item Can I list at least 4 technologies that were developed as a result of
			space exploration?
		\item Can I describe how some of the technologies developed as a result of 
			space exploration impact our daily lives?
		\item Can I list some of hte benefits associated with space exploration?
		\item Can I show by calculation that the same mass of different materials
			requires different quantities of energy to raise their temperature
			by 1 degree Celsius per unit mass?
		\item What is meant by the term 'specific heat capacity'?
		\item What is meant by the term 'change of state'?
		\item What is meant by the term 'specific latent heat of vaporisation'?
		\item Can I carry out calculations involving heat, mass, specific heat
			capacity and temperature change fora spacecraft during re-entry?
		\item What happens to the temperature of a substance when it changes state?
		\item Can I carry oyt calculations involving heat, mass and specific latent
			heat for a spacecraft during re-entry?
		\item What are the challenges faced by a space craft when re-entering a
			planet's atmosphere?
		\item Can I identify which materials could be used on the thermal protection
			system on a space craft to protect it on re-entry and state why they should
			be used.
		\item Can I describe the need for thermal protection systems to protect
			a spacecraft during re-entry?
		\item Can I describe the challenges of re-entry to the Earth's atmosphere?
		\item Can I list some of the risks associated with space exploration?

	\end{itemize}
\end{multicols}
